\documentclass[smallheadings]{scrartcl}

%%% GENERAL PACKAGES %%%%%%%%%%%%%%%%%%%%%%%%%%%%%%%%%%%%%%%%%%%%%%%%%%%%%%%%%%
% inputenc allows the usage of non-ascii characters in the LaTeX source code
\usepackage[utf8]{inputenc}
\usepackage{graphicx}


% title of the document
\title{LGS mit LU Zerlegung}
% optional subtitle
%\subtitle{Draft from~\today}
% information about the author
\author{%
 Aron Ventura und Annika Thiele\\ Humboldt-Universit\"at zu Berlin
}
\date{\today}


%%% LANGUAGE %%%%%%%%%%%%%%%%%%%%%%%%%%%%%%%%%%%%%%%%%%%%%%%%%%%%%%%%%%%%%%%%%%
% babel provides hyphenation patterns and translations of keywords like 'table
% of contents'
\usepackage[ngerman]{babel}

\usepackage{amsthm}
\newtheorem{theorem}{Satz}


\theoremstyle{definition}
\newtheorem{definition}{Definition}[section]


%%% HYPERLINKS %%%%%%%%%%%%%%%%%%%%%%%%%%%%%%%%%%%%%%%%%%%%%%%%%%%%%%%%%%%%%%%%
% automatic generation of hyperlinks for references and URIs
\usepackage{hyperref}

%%% MATH %%%%%%%%%%%%%%%%%%%%%%%%%%%%%%%%%%%%%%%%%%%%%%%%%%%%%%%%%%%%%%%%%%%%%%
% amsmath provides commands for type-setting mathematical formulas
\usepackage{amsmath}
% amssymb provides additional symbols
\usepackage{amssymb}
% HINT
% Use http://detexify.kirelabs.org/classify.html to find unknown symbols!

%%% COLORS %%%%%%%%%%%%%%%%%%%%%%%%%%%%%%%%%%%%%%%%%%%%%%%%%%%%%%%%%%%%%%%%%%%%
% define own colors and use colored text
\usepackage[pdftex,svgnames,hyperref]{xcolor}

%%% nice tables
\usepackage{booktabs}

%%% Code Listings %%%%%%%%%%%%%%%%
% provides commands for including code (python, latex, ...)
\usepackage{listings}
\definecolor{keywords}{RGB}{255,0,90}
\definecolor{comments}{RGB}{0,0,113}
\definecolor{red}{RGB}{160,0,0}
\definecolor{green}{RGB}{0,150,0}
\lstset{language=Python, 
        basicstyle=\ttfamily\small, 
        keywordstyle=\color{keywords},
        commentstyle=\color{comments},
        stringstyle=\color{red},
        showstringspaces=false,
        identifierstyle=\color{green},
        }

\usepackage{graphicx}
\usepackage{paralist}

\usepackage[style=authoryear, backend=biber,natbib=true]{biblatex}
\addbibresource{LU}

% setting the font style for input und returns in description items
\newcommand{\initem}[2]{\item[\hspace{0.5em} {\normalfont\ttfamily{#1}} {\normalfont\itshape{(#2)}\/}]}
\newcommand{\outitem}[1]{\item[\hspace{0.5em} \normalfont\itshape{(#1)}\/]}
\newcommand{\bfpara}[1]{
	
	\noindent \textbf{#1:}\,}

\begin{document}

% generating the title page
\maketitle
\newpage
% generating the table of contents (requires to run pdflatex twice!)
\tableofcontents
\newpage

%%% BEGIN OF CONTENT %%%%%%%%%%%%%%%%%%%%%%%%%%%%%%%%%%%%%%%%%%%%%%%%%%%%%%%%%%

\section{Einf\"uhrung in die Theorie}
	\subsection{Motivation und Problemstellung}
		Wir beschäftigen uns in diesem Bericht weiter mit dem numerischen Lösen des 
		Poisson Differentialgleichungssystem.  Wir haben bereits gesehen,  dass wir das 
		Poisson Problem numerisch lösen können, indem wir das Gebiet und den 
		Laplaceoperator diskretisieren und so ein lineares Gleichungssystem erhalten.  Um 
		eine gute Approximation zu berechnen,  möchten wir möglichst fein diskretisieren.  
		Mit zunehmender Feinheit,  wächst allerdings auch die Größe der Matrix,  welche 
		das lineare Gleichungssystem darstellt.  Allerdings sind viele Einträge Null. Man 
		spricht hierbei auch von einer sparsen Matrix. Dabei ist unsere bisheriger Ansatz,  
		das Problem mit der LU-Zerlegung zu lösen nicht sehr effizient.  Wir haben 
		gemerkt,  dass die LU-Zerlegung die Sparsität der Gleichungssysteme nicht 
		beibehält. Daher stellt sich die Frage,  wie wir große große (dünnbesetzte) 
		Gleichungssysteme lösen können und die Sparsität für diesen Zweck ausnützen 
		können.  In diesem Bericht beschäftigen wir uns dafür mit einem iterativen 
		Verfahren und vergleichen dieses mit dem Verfahren der LU-Zerlegung. Wir 
		ziehen dabei sowohl den Speicherbedarf als auch den Rechenaufwand in 
		Betracht. 
	\subsection{Theoretischer Hintergrund}
		\subsubsection{Iterative numerische Verfahren}
			Im Vergleich zu dem Verfahren zum lösen linearer Gleichungssysteme 
			mit der LU-Zerlegung,  enden iterative numerische Verfahren nicht nach
			endlich vielen Schritten mit einer exakten Lösung, sonder nähern sich vielmehr 
			Schrittweise an die genaue Lösung an.  Wie wir gesehen haben, kommt es 
			auch bei dem exakten Lösen des Problems zu Fehlern.  Daher ist es sinnvoll
			sich anzuschauen, ob man mit iterativen Verfahren mit weniger Aufwand
			und Speicherbedarf zu einer ähnlich gut approximierten Lösung
			kommen kann.  Dass ein solches Iterationsverfahren konvergiert auf
			Grundlage des Banachschen Fixpunktsatzes.
			Wir definieren zunächst wie in \citep{kont}
			\begin{definition}[Kontraktion]
			Sei $(X,d)$ ein metrische Raum.  Eine Abbildung $f:X\rightarrow X$ heißt 
			\textit{Kontraktion}, wenn sie Lipschitz stetig ist mit einer Lipschitzkonstanten
			$0\leq \lambda < 1$,  es also für alle $x,y\in X$ gilt:
			$$d(f(x),f(y))\leq \lambda d(x,y)$$
			\end{definition}
			Die Aussage des folgenden Satzes beruht auf \citep{ban}
			\begin{theorem}[Banachscher Fixpunktsatz]
			Sei $(X; d)$ ein vollständiger Metrischer Raum.  Sei $A\subset X$ 
			abgeschlossen und nicht-leer. Sei $f:A\rightarrow A$ eine Kontraktion. Definiere
			eine Folge $(x_n)$ mit beliebigen Startwert durch 
			$$x_{n+1}=f(x_n)$$
			Dann existiert genau ein Fixpunkt von $f$.  Dieser ist der Grenzwert von 
			$(x_n)$.
			\end{theorem}
			Wir betrachten ein lineares Gleichungssystem $Ax=b$ für eine Matrix
			$A\in \mathbb{R}^{m\cross m}$ Unser Iterationsansatz von \citep{skript}
			 $$x_{n+1}=Bx_n+d$$
			 für geeignet gewähltes $B$ und $d$,  konvergiert nach dem Banachschen 
			 Fixpunktsatz, wenn die rechte Seite als Abbildung eine 
			 Kontraktion ist. 
			Wir werden uns in diesem Bericht ein spezielles Splitting Verfahren genauer
			anschauen.  Allgemein zerlegt man bei Splitting Verfahren wie in \citep{skript}
			beschrieben die Matrix $A=M+N$ in eine Summe von Matrizen und wählt dann
			$$B:=-M^{-1}N,  \text{ } d=M^{-1}b.$$
			Es wird in \citep{skript} gezeigt, dass die Iteration mit genau dann eine
			Kontraktion ist und somit konvergiert, wenn 
			$\rho (B):=\max_{1\leq i \leq m} |\lambda _i(B)|<1$ gilt.
			Wir betrachten folgende Zerlegung von $A$ in die Summe einer unteren
			Dreiecksmatrix, einer Diagonalmatrix und einer oberen Dreiecksmatrix.
			\begin{definition}
			Sei $A\in \mathbb{R}^{m\cross m}$.  Wir definieren die Matrizen
			$$A=L+D+U:=\begin{pmatrix}
			0&0&\hdots &0\\
			*&0&\hdots &0\\
			\vdots &&\ddots &\vdots\\
			*&*&\hdots &0
			\end{pmatrix}+\begin{pmatrix}
			*&0&\hdots &0\\
			0&*&\hdots &0\\
			\vdots &&\ddots &\vdots\\
			0&0&\hdots &*
			\end{pmatrix}+\begin{pmatrix}
			0&*&\hdots &*\\
			0&0&\hdots &*\\
			\vdots &&\ddots &\vdots\\
			0&0&\hdots &0
			\end{pmatrix}$$
			
			\end{definition}
			
			Dann betrachten wir für das sogenannte \textbf{SOR-Verfahren (Successive 
			Over Relaxation} wie auch in \citep{skrpit} die Zerlegung  mit
			$M:=L+\frac{1}{\omega}$ und $N:=U-(\frac{1}{\omega} -1)D$,  also die 
			Iteration 
			\begin{align}\label{sor_iteration}
			x_{n+1}=\omega (-D^{-1}Ux_n-D^{-1}Lx_{n+1}+D^{-1})+(1-\omega )x_n
			\end{align}
			für ein geeignetes $\omega$.
			
			Da wir bei iterativen Verfahren nicht nach endlich vielen Schritten zu einer
			exakten Lösung kommen,  möchten wir das Verfahren bei einer gewissen
			Genauigkeit der Lösung abbrechen.  Dafür geben wir einen Wert $\varepsilon$ 
			an, und setzen die Abbruchbedingung als $$||Ax_i-b||<\varepsilon$$ für $x_i$ 
			als Lösung des $i$-ten Iterationsschrittes. 
			
		\subsubsection{Rechenaufwand und Speicherplatz}	
		
		Im Gegensatz zum Verfahren mit der 
		LU-Zerlegung,  benötigen wir in dem oben beschriebenen Iterativen Verfahren 
		nicht mehr Speicherplatz, denn die Anzahl an Nicht-Null Eintägen der Zerlegung 
		ist genauso groß wie die der Matrix $A$.
		
		Der Rechenaufwand von dem SOR-Verfahren ist in jedem Schritt nach
		 \citep{slides} linear beschränkt, also in $\mathcal{O}(n)$, für eine 
		 dünnbesetzte Matrix $A\in\mathbb{R}^{n\cross nm}$
		 
		 \subsubsection{Optimaler Wert $\omega$}
		 
\section{Implementierung}
\section{Beschreibung der Experimente}
\section{Auswertung der Experimente}
\section{Zusammenfassung}



\printbibliography

%%% END OF DOCUMENT %%%%%%%%%%%%%%%%%%%%%%%%%%%%%%%%%%%%%%%%%%%%%%%%%%%%%%%%%%%
\end{document}
